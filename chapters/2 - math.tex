 \chapter{Literature Review}

Substantial literature has been published that aim to evaluate and compare the performance of algorithms using emperical evidiencce. In \cite{adam}, a frequently cited paper, the author designed and conducted experiments where metrics such as cost per iteration of commonly used algorithms are recorded. The result gives a general idea of different algorithms' characteristic solving different problems.

There exist in the literature many approaches to deriving a performance bound of first-order optimization algorithms. In 2014, Drori and Teboulle first introduced the method of representing a class of function with constraints, reformulating the problem of analyzing an optimization method into a semidefinite program (SDP) whose size is proportionate with the number of iterations the algorithm is run.\cite{drori2012}. The paper coined the term Performance Estimation Problem (PEP) and showed that by solving convex semidefinite problem, a worst-case numerical bound on an algorithm's performance solving that class of function can be derived. Taylor, Hendrickx and Glineur built upon this work by introducing the ideas of creating a finite representation for a class of smooth strongly convex functions using closed-form necessary and sufficient conditions. While the above mentioned two approaches give the performance bound in the form of a guarantee how close \(x_k\) is to the goal \(x_*\) after a fixed number of iterates, the method used in this package proves that the performance measure inputted by the users decreases at a guaranteed rate throught out the optimizing process.

IQCs here

Of the methodologies above, PEP has been implemented into a computer program in PESTO, a MATLAB toolbox and PEPit, a Python package. The program when given a first-order method, a class of function, a performance measure, and an initial condition, will find the worst-case performance automatically. PESTO and PEPit follows the PEP methodology and presented an easy way to analyze gradient-based algorithms.

The main contribution of this thesis paper is to create an alternative computer program similar PESTO and PEPit that aims to provide an accessible and fast way to analyze the performance of first-order methods for a guaranteed convergence rate, leveraging the approach presented in \cite{tutorial}, in the Julia programming language.