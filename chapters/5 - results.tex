\chapter{Analysis Process and Result}\label{chapter:result}

The Lyapunov function approach require 2 components which forms the semidefinite problem described in \Cref{chapter:lyapunov} to produce a worst-case performance convergence rate: the state update matrices and the constraints formed by the interpolation conditions. These components are derived from the input provided to the program which undergo transformations before they can be used to create an optimization problem in the JuMP modelling language, the process of which can be described in 3 steps:
\begin{enumerate}
    \item The Algorithm Analysis program automatically uses the input provided to form a systematic charaterization the analysis problem using data structures described in \Cref{chapter:code}, including how the algorithm being analyzed updates, the constraints created by the interpolation conditions of the class of function and the performance measure.
    \item These data structures are converted to real number vectors and matrices that represent the updated state of the algorithm and each algorithm in the form of the a linear function of the initial states and inputs.
    \item An optimization problem is created inside a JuMP model using these representations and solved to verify whether a certain convergence rate is feasible for a given problem. This process is repeated with different convergence rates as the program search for the lowest feasible convergence rate.
\end{enumerate}

This chapter details the analysis process of analyzing gradient descent's performance at optimizing any $1$-$10$ sector bounded function as shown in Figure~\ref{ex_analysis}, including how these steps presented in \Cref{chapter:lyapunov} are performed and how the optmization problem is formed and solved to derive worst-case performancce convergence rate.

\subsection*{Real scalars and linear form}

% In JuPE, when variable expressions are defined in an inner product space, they are vector. These expressions include states of an algorithm, the starting and end points, and the gradient of a function at a point. This is because a function class is multidimensional, which means each point or iterate is a vector whose size is identical to the dimension of the function. 

% All three components needed to form the linear matrix inequality - the Lyapunov functions $V(x_k)$ and $V(x_{k+1})$ and the left hand side of each constraint - from equations \ref{eqn:int_cond2}, \ref{eqn:Ly_ineq} and \ref{eqn:Ly_ineq2} are functions linear in the elements of the Gram matrix and optimization variables. On the other hand, the JuMP modeling language does not support the expression and constraint data structures presented in chapter 4, and the LMIs of \ref{Ly_ineq2} must be transformed into a function linear in the optimization variables and real numbers before it can be formed inside the JuMP model. 
As specified in \ref{Lyapunov}, the linear matrix inequalities are constructed from the linear form of the Lyapunov functions and the constraints as a function of the vector $[x; u]$. This process is done in three steps, which are:
\begin{enumerate}
    \item Of every expressions that has been created during the input process, define the initial state vector \texttt{x} as every real expressions which contain another expression in its next field.
\begin{figure}[h!]
    \begin{lstlisting}[mathescape]
x  = collect(v for v $\in $ vars if !ismissing(next(v)) && v isa R)
4-element Vector{R}:
    |x0|$^2$
    |xs|$^2$
    <x0,xs>
    <xs,x0>
\end{lstlisting}
\caption{Initial state real scalar expressions from example \ref{ex_analysis}}
\label{ex_initstate}
\end{figure}

    \item Define an update state vector x$^+$ consisting of the next expression of every expression in the initial state. The input vector u is then defined as every real expression that exist in the decomposition of the updated state expressions but not in the initial state expressions.
    \begin{figure}[h!]
        \begin{lstlisting}[mathescape]    
x$^+$ = next(x)
4-element Vector{R}:
    <x0,xs> - 0.18181818181818182 <$\nabla $f(x0),xs>
    -0.18181818181818182 <$\nabla $f(x0),x0> + 0.03305785123966942 |$\nabla $f(x0)|$^2$ - 0.18181818181818182 <x0,$\nabla $f(x0)> + |x0|$^2$
    -0.18181818181818182 <xs,$\nabla $f(x0)> + <xs,x0>
    |xs|$^2$

u  = collect(setdiff(variables(x$^+$), variables(x)))
5-element Vector{Expression}:
    <xs,$\nabla $f(x0)>
    <x0,$\nabla $f(x0)>
    <$\nabla $f(x0),xs>
    <$\nabla $f(x0),x0>
    |$\nabla $f(x0)|$^2$
        \end{lstlisting}    
    \caption{Updated state and input real scalar expressions from example \ref{ex_analysis}}
    \label{ex_updatedstate_input}
    \end{figure}
    \item The initial state and input vector \texttt{[x; u]} is the code equivalent of $\bmat{x & u}^\tp$ and can form every expressions required to form the linear matrix inequality. This transformation, which will be refered to as the linear form of an expression, can be derived by finding the values of each expression in the vector \texttt{[x; u]} present in the expression's decomposition dictionary.
\end{enumerate}

\begin{figure}[h!]
    \begin{lstlisting}[mathescape] 
linear_form = vec(linearform([x; u] => x0^2 - 3*(x0'*xs)))
linear_form'*[x; u]
Scalar in R
  Decomposition: -3 <xs,x0> + |x0|$^2$
\end{lstlisting}    
\caption{Example of linear form of a scalar expression \ref{ex_analysis}}
\label{ex_linearform}
\end{figure}

\section{Performance measure}
The linear form matrix of the performance measure is the first of the three components needed to form the Lyapunov function ($\|x_k - x_s\|^2$ in \eqref{eqn:Ly_ineq}). For example, the performance measure in \ref{ex_analysis}, which is defined as $(x0-xs)^2$ and which evaluates into $|x0|^2 - <xs, x0> - <x0, xs> + |xs|^2$, has the linear form presented in Figure~\ref{ex_linearform2}.
\begin{figure}[h!]
\begin{lstlisting}[mathescape]
$\mathcal{P} $ = vec(linearform( [x; u] => performance ))
print($\mathcal{P}$)
[-1, -1, 1, 1, 0, 0, 0, 0, 0]
\end{lstlisting}
\caption{Linear form matrix of expression $(x0-xs)^2$}
\label{ex_linearform2}
\end{figure}

\section{Algorithm, state update and Lyapunov function formulation}
% \subsection*{State space matrix and their linear form formulation}
As shown in Figure~\ref{ex_analysis} and in section \ref{states}, the algorithm to be analyzed is provided as input first by defining an initial state and how the next state is updated from the initial state. The initial state is defined to be a vector in an inner product space, and the updated state is a linear function of one or multiple initial state and the gradient of the function evaluated at some point. While the gradient descent algorithm updates using only one state and evaluate the gradient at the previous state, if an algorithm updates using multiple past states or the gradient at an interpolated point, these vectors will also have to be defined. The forming of the algorithm can then be completed by defining the relationship between states and their next states using the \texttt{=>} operation, which updates the next field of every expression in the decomposition of which there is the state on the left hand side of the operation. For example, in Figure~\ref{ex_analysis}, the state vector \texttt{x1} is defined as a function of the state vector \texttt{x0} and as the next state of \texttt{x0}, while the next state of the stationary point \texttt{xs} is itself. This not only means the \texttt{next} field of \texttt{x0} and \texttt{xs} are \texttt{x1} and \texttt{xs} respectively, but also next state of any every expression derived from the norm, inner product, or algebraic calculation of which \texttt{x0} is a part is that calculation done with \texttt{x1} instead. In this example, as shown in Figure~\ref{ex_next}, the next state of the inner product of $x_0$ and $x_s$ denoted as \texttt{next(x0'*xs)} is the inner product of $x_1$ and $x_s$ \texttt{x1'*xs}. This enables the operation in Figure~\ref{ex_linearform_next_state} and allow any updated iterate to be automatically expressed a linear function of the intial states and inputs.
\begin{figure}[h!]
	\begin{lstlisting}[mathescape]
next(x0)

Vector in R$^n$
  Label: x1
  Decomposition: -0.18181818181818182 $\nabla $f(x0) + x0
  Associations: Dual => x1*

next(x0'*xs)

  Scalar in R
    Decomposition: -0.18181818181818182 <xs,$ \nabla $f(x0)> + <xs,x0>
\end{lstlisting}
\caption{\texttt{next} field of state a vector expression and a scalar formed from a state expression}
\label{ex_next}
\end{figure}

As the initial state vector is defined in Figure~\ref{ex_initstate} and the updated state vectors in Figure~\ref{ex_updatedstate_input}, their linear form matrices is the second of the three components needed to formulate the Lyapunov function and can be formed as shown in Figure~\ref{ex_linearform_state} and Figure~\ref{ex_linearform_next_state}.

\begin{figure}[h!]
    \begin{lstlisting}[mathescape]
X  = linearform([x; u] => x)
    4x9 Matrix{Int64}:
    1  0  0  0  0  0  0  0  0
    0  1  0  0  0  0  0  0  0
    0  0  1  0  0  0  0  0  0
    0  0  0  1  0  0  0  0  0    
    \end{lstlisting}
    \caption{Linear form state matrix x}
    \label{ex_linearform_state}
\end{figure}
\begin{figure}[h!]
    \begin{lstlisting}[mathescape]
X$^+$ = linearform([x; u] => x$^+$)
4x9 Matrix{Real}:
1  0  0  0   0        0       -0.1818   0         0
0  1  0  0  -0.1818   0        0        0.03306  -0.1818
0  0  1  0   0       -0.1818   0        0         0
0  0  0  1   0        0        0        0         0
\end{lstlisting}
\caption{Linear form state matrix x$^+$}
\label{ex_linearform_next_state}
\end{figure}
% \subsection*{State space in Lyapunov function}
Following the steps presented in chapter 3, the Lyapunov function can begin to be formed by first defining an optimization variable $P$ in the JuMP model as a JuMP variable. Once JuMP and the solver start optimizing the problem, P is one of the variable that will be optimized to produce a solution. The Lyapunov functions are then created following \eqref{eqn:Ly_ineq} but with code variables as:

\begin{subequations} \label{eqn:Ly_ineq3}
	\begin{align}
	    L1 = \mathcal{P} - X^\tp P     \\
        L2 = X^{+\tp} P - \rho X^\tp P
	\end{align}
\end{subequations}

\section{Constraints}
As presented in \ref{oracles}, the oracle created from the class of function and the transpose of each expression automatically forms the interpolation condition and Gram matrix constraints. These constraints are linearized and added to the optimization in 2 steps:
\begin{description}
    \item [Optimization variable multipliers] For each constraint $i \in$, two optimization variables $\lambda_i$ and $\mu_i$, which represent $\Lambda^1_i$ and $\Lambda^2_i$ in \Cref{chapter:lyapunov} are defined as JuMP variables. If the constraint is applied to the Gram matrix, the optimization variable will be a matrix sharing the same size with the matrix constrained. Otherwise, if the constraint is created from the interpolation conditions of the class of function and is applied to a single real scalar expression, the optimization problem created will have a size of 1.
    \item [Constraint on multiplier] The JuMP variables multipliers are constrained in the JuMP model depending on the constraint expression they were created for: The multiplier is not constrained if the expression is constrained to be zero, constrained to be non-negative if the expression is constrained to be non-negative, and constrained to be symmetrical and in the JuMP supported positive semidefinite cone if the expression is constrained to be positive semidefinite.
    \item [Linear form of constraints] The linear form of each constraint scaled by the multiplier is created and added to the Lyapunov functions.
\end{description}

If the expression constrained is a single real scalar, the linear form of the constraint is derived similarly to the linear form of the performance measure or state space matrices but scaled by the multiplier. Suppose we have the constraint $(x0 - xs)^2 \geq 0$ and matrix $\bmat{x\\ u} = \bmat{|x0|^2\\ <xs, x0>\\ |xs|^2}$, the linear form of the constraint in terms of $\bmat{x\\ u}$, denoted as $M$ would be:

\begin{subequations} \label{eqn:constraint_single}
	\begin{align}
    \lambda * (x0 - xs)^2 &= M * \bmat{|x0|^2\\ <xs, x0>\\ |xs|^2} \label{eq_cons_single1}       \\
	M &= \bmat{\lambda& 2\lambda & \lambda} \label{eq_cons_single2}
	\end{align}
\end{subequations}

If the expression constrained and its corresponding multiplier are vectors of expression, the linear form of the constraint is derived as the linear form of the inner product between the multiplier vector and the constraint expression vector. Suppose we have a constraint vector $\bmat{(x0-xs)^2 \\ (x0-xs)^2-3*|xs|^2} \geq 0$ and the same $\bmat{x\\ u}$ matrix as \eqref{eqn:constraint_single}, the linear form of the constraint in terms of $\bmat{x\\ u}$, denoted as $M$ would be:

\begin{subequations} \label{eqn:constraint_vector}
	\begin{align}
    \bmat{\lambda  & \lambda } * \bmat{(x0 - xs)^2 \\ (x0 - xs)^2-3*|xs|^2} &= M * \bmat{|x0|^2\\ <xs, x0>\\ |xs|^2} \label{eq_cons_vector1}       \\
	M &= \bmat{\lambda & -\lambda & \lambda \\ \lambda & -\lambda & -2\lambda} \label{eq_cons_vector2}
	\end{align}
\end{subequations}

And if the expression constrained and its corresponding multiplier are matrices, the linear form of the constraint is the linear form of the trace of the matrix multiplication between the multiplier and the constraint expression. For the Gram matrix in \eqref{eqn:trans_cond} which is constrained to be positive semidefinite, its linear form would be:

\begin{equation} \label{eqn:trans_cond}
	tr(\lambda \bmat{||x0||^2 & \innerproduct{xs}{x0} & \innerproduct{\nabla f(x0)}{x0} \\ \innerproduct{x0}{xs} & ||xs||^2 & \innerproduct{\nabla f(x0)}{xs} \\ \innerproduct{x0}{\nabla f(x0)} & \innerproduct{xs}{\nabla f(x0)} & ||\nabla f(x0)||^2]}) \geq 0	
\end{equation}

Where $\lambda $ is a 3x3 JuMP variable. In all three cases, for each constraints, 2 identical linear form matrices are created, one scaled by \texttt{$\lambda$} and added to the first Lyapunov function and the other by \texttt{$\mu$} and added to the second Lyapunov function. This completes the final linear matrix inequalities as defined in \eqref{eqn:LMI1}.

\section{Derived feasibility and bisection search}
Upon the completion of the linear matrix inequalities, the solver of the JuMP model is called to optimize the problem and find the variables \texttt{$P$}, \texttt{$\lambda$}s and \texttt{$\mu$}s for which the linear matrix inequality is satisfied and a convergence rate $\rho$ can be guaranteed.

Using the definition of the convergence rate in \eqref{eqn:convergence_rate}, a \texttt{$\rho$} value of 1 means the algorithm cannot be guaranteed to converge and a convergence rate of 0 means the algorithm is guaranteed to converge after a single iterate. In order to find the worst-case performance rate, the program performs bisection search, also known as binary search, for the smallest value $\rho$ between 0 and 1 that makes the optimization problem feasible, calling a function to perform the steps presented in this chapter for each value \texttt{$\rho$} and checking feasibility at each iterate of the search. The smallest value \texttt{$\rho$} found within a tolerance of $10^{-5}$ is returned as the guaranteed convergence rate, and the analysis process is complete.