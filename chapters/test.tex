\documentclass{article}

\usepackage[outputdir=build]{minted}
\usepackage{textalpha}
\usepackage[capitalize]{cleveref}
\usepackage{tcolorbox}
\usepackage{xcolor}
\usepackage{geometry}

\DeclareUnicodeCharacter{207F}{\textsuperscript{n}}
\DeclareUnicodeCharacter{2208}{\ensuremath{\in}}

\newenvironment{code}[4][]
 {\VerbatimEnvironment
  \begin{listing}
  \caption{#3}\label{#4}
  \begin{tcolorbox}
    \begin{minted}[
    linenos,
    fontsize=\footnotesize,
    xleftmargin=21pt,
    baselinestretch=1,
    tabsize=4,
    #1]{#2}}
 {\end{minted}\end{tcolorbox}\end{listing}}


\begin{document}

Our package enables users to write optimization algorithms using simple syntax that resembles the mathematical algorithm structure, as illustrated in \cref{code:label}.

\begin{code}{julia}{A sample code listing. The method \texttt{rate} performs the automated algorithm analysis by searching for a Lyapunov function that bounds the rate of convergence of the performance measure.}{code:label}
  m,L = 1,10                                   # parameters
  α = 2/(L+m)                                  # stepsize
  
  @algorithm begin
    f = DifferentiableFunctional{Rⁿ}()       # objective function
    xs = first_order_stationary_point(f)     # first-order stationary point
    f' ∈ SectorBounded(m, L, xs, f'(xs))     # function class
    x0 = Rⁿ()                                # initial point
    x1 = x0 - α*f'(x0)                       # gradient descent
    x0 => x1                                 # state update
    performance = (x1-xs)^2                  # performance measure
  end
  
  rate(performance)                            # automated algorithm analysis
  \end{code}
  

\end{document}