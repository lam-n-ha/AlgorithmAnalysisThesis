\let\origtime\time
\documentclass{thesis}
\let\time\origtime
% Class options: [singlespacing, onehalfspacing]

%%%%%%%%%%%%%%%%%%%%%%%%%%%%%%%%%%%%%%%%%%%%%%%%%%%%%%%%%%%%%%%%%%%%%%%%%%%%%%%
%% BASIC INFORMATION
%%%%%%%%%%%%%%%%%%%%%%%%%%%%%%%%%%%%%%%%%%%%%%%%%%%%%%%%%%%%%%%%%%%%%%%%%%%%%%%
\name{Lam Ngoc}{Ha}
\title{Algorithm Analysis: Julia\texttrademark Package Implementation of Optimization Algorithm Analysis Using Lyapunov Function}
\school{Miami University}
\college{Engineering and Computing}
\department{Electrical and Computer Engineering}
\location{Oxford}{Ohio}
\year{2024}

% \advisor{Dr. Advisor}
\listadd{\advisors}{Dr. Bryan Van Scoy}
\listadd{\committee}{Dr. Veena Chidurala}
\listadd{\committee}{Dr. Peter Jamieson}


%%%%%%%%%%%%%%%%%%%%%%%%%%%%%%%%%%%%%%%%%%%%%%%%%%%%%%%%%%%%%%%%%%%%%%%%%%%%%%%
%% PACKAGES (not required)
%%%%%%%%%%%%%%%%%%%%%%%%%%%%%%%%%%%%%%%%%%%%%%%%%%%%%%%%%%%%%%%%%%%%%%%%%%%%%%%
\usepackage{lipsum}                   % filler text
\usepackage{graphicx}                 % figures
\usepackage{amsmath,amssymb,amsthm}   % mathematics
\usepackage{colonequals}              % for special := and =: symbols
\usepackage[shortlabels]{enumitem}    % customizable itemization
\usepackage{cite}                     % citation shortening
\usepackage{calc}                     % allows arithmetic with LaTeX lengths
\usepackage{booktabs}                 % pretty tables
\usepackage{multirow}
\usepackage{siunitx}
\usepackage{xcolor}
\usepackage{parskip}
\usepackage{acronym}
\usepackage{thmtools}
\usepackage{listings}
\usepackage{svg}
\usepackage[capposition=bottom]{floatrow}

\usepackage{subcaption}
% \usepackage{minted}
% \usepackage{textalpha}
% \usepackage{tcolorbox}
% \usepackage{geometry}
% \usepackage[margin=0.25in]{geometry}

% customized figure captions
\usepackage[margin=10pt,font=small,labelfont=bf,labelsep=colon]{caption}
\captionsetup[figure]{name=Figure}
\captionsetup[table]{aboveskip=3pt}

% clever references (also for theorems and such)
\usepackage[capitalise,nameinlink]{cleveref}

\usepackage[normalem]{ulem}
% \newcommand{\comment}[1]{\textcolor{blue}{#1}}
% !TeX TXS-program:compile = txs:///pdflatex/[--shell-escape]

\lstdefinelanguage{Julia}%
  {morekeywords={abstract,break,case,catch,const,continue,do,else,elseif,%
      end,export,false,for,function,immutable,import,importall,if,%
      macro,module,otherwise,quote,return,switch,true,try,type,typealias,%
      using,while, @show, rate, @algorithm, DifferentiableFunctional, first_order_stationary_point,
      SectorBounded, R$ ^n $},%
   sensitive=true,%
   morecomment=[l]\#,%
   morecomment=[n]{\#=}{=\#},%
   morestring=[s]{"}{"},%
   morestring=[m]{'}{'},%
}[keywords,comments,strings]%

% \DeclareUnicodeCharacter{207F}{\textsuperscript{n}}
% \DeclareUnicodeCharacter{2208}{\ensuremath{\in}}

% \newenvironment{code}[4][]
%  {\VerbatimEnvironment
%   \begin{listing}
%   \caption{#3}\label{#4}
%   \begin{tcolorbox}
%     \begin{minted}[
%     linenos,
%     fontsize=\footnotesize,
%     xleftmargin=21pt,
%     baselinestretch=1,
%     tabsize=4,
%     #1]{#2}}
%  {\end{minted}\end{tcolorbox}\end{listing}}

\lstset{%
    language         = Julia,
    breaklines       = true,
    %postbreak        = \mbox{\textcolor{red}{$\hookrightarrow$}\space},
    basicstyle       = \ttfamily,
    keywordstyle     = \bfseries\color{blue},
    stringstyle      = \color{magenta},
    commentstyle     = \color{ForestGreen},
    showstringspaces = false
}

\theoremstyle{definition}
\declaretheorem[name=Theorem, Refname={Theorem,Theorems}]{theorem}
\declaretheorem[name=Lemma, Refname={Lemma,Lemmas}, sibling=theorem]{lemma}
\declaretheorem[name=Corollary, Refname={Corollary,Corollaries}, sibling=theorem]{corollary}
\declaretheorem[name=Proposition, Refname={Proposition,Propositions}, sibling=theorem]{proposition}
\declaretheorem[name=Definition, Refname={Definition,Definitions}, sibling=theorem]{definition}
\declaretheorem[name=Theorem, Refname={Theorem,Theorems}, sibling=theorem,
			    shaded={bgcolor=black!20,margin=1ex,textwidth=\linewidth-2ex}]{theoremshaded}
\declaretheorem[name=Theorem, Refname={Theorem,Theorems}, sibling=theorem,
	            shaded={rulecolor=black, bgcolor=white, rulewidth=1pt, margin=1ex, textwidth=\linewidth-2ex-2pt}]{theoremboxed}

% more legible proof environment and QED symbol
\def\qed{\rule[0pt]{5pt}{5pt}\par\medskip}
\renewcommand{\qedhere}{\hfill ~\qed}
\newcommand{\innerproduct}[2]{\langle #1, #2 \rangle}
\renewenvironment{proof}{{\noindent\bf Proof.}}{\qedhere}


% automatically look for graphics in these folders
\graphicspath{{graphics/}}


%%%%%%%%%%%%%%%%%%%%%%%%%%%%%%%%%%%%%%%%%%%%%%%%%%%%%%%%%%%%%%%%%%%%%%%%%%%%%%%
%% DEFINITIONS (not required)
%%%%%%%%%%%%%%%%%%%%%%%%%%%%%%%%%%%%%%%%%%%%%%%%%%%%%%%%%%%%%%%%%%%%%%%%%%%%%%%
\def\integer{\mathbb{Z}}                     % integers
\def\real{\mathbb{R}}                        % real numbers
\def\complex{\mathbb{C}}                     % complex numbers
\def\tp{\mathsf{T}}                          % tranpose
\def\Re{\mathrm{Re}}                         % real part of a complex number
\def\Im{\mathrm{Im}}                         % imaginary part of a complex number
\def\epsilon{\varepsilon}                    % epsilon
\def\defeq{\colonequals}                     % definitions
\def\eqdef{\equalscolon}                     % definitions
\def\grad{\nabla}                            % gradient
\DeclareMathOperator*{\argmin}{\arg\min}     % arg min
\DeclareMathOperator*{\argmax}{\arg\max}     % arg max
\DeclareMathOperator*{\minimize}{minimize}   % min
\DeclareMathOperator*{\maximize}{maximize}   % max
\DeclareMathOperator{\trace}{\mathrm{tr}}    % trace
\DeclareMathOperator{\diag}{\mathrm{diag}}   % diag

% MATRICES
\newcommand{\bmat}[1]{\begin{bmatrix}#1\end{bmatrix}}
\newcommand{\pmat}[1]{\begin{pmatrix}#1\end{pmatrix}}



%%%%%%%%%%%%%%%%%%%%%%%%%%%%%%%%%%%%%%%%%%%%%%%%%%%%%%%%%%%%%%%%%%%%%%%%%%%%%%%
%% MAIN DOCUMENT
%%%%%%%%%%%%%%%%%%%%%%%%%%%%%%%%%%%%%%%%%%%%%%%%%%%%%%%%%%%%%%%%%%%%%%%%%%%%%%%
\begin{document}

\frontmatter

%%%%%%%%%%%%%%%%%%%%%%%%%%%%%%%%%%%%%%%%%%%%%%%%%%%%%%%%%%%%%%%%%%%%%%%%%%%%%%%
%% TITLE AND SIGNATURE PAGE
%%%%%%%%%%%%%%%%%%%%%%%%%%%%%%%%%%%%%%%%%%%%%%%%%%%%%%%%%%%%%%%%%%%%%%%%%%%%%%%
\maketitle

%%%%%%%%%%%%%%%%%%%%%%%%%%%%%%%%%%%%%%%%%%%%%%%%%%%%%%%%%%%%%%%%%%%%%%%%%%%%%%%
%% ABSTRACT (200 word max, one paragraph)
%%%%%%%%%%%%%%%%%%%%%%%%%%%%%%%%%%%%%%%%%%%%%%%%%%%%%%%%%%%%%%%%%%%%%%%%%%%%%%%
\begin{abstract}
  Many problems that appear in science and engineering can be considered optimization problems, and first-order iterative algorithms are used to solve large scale problems in fields such as machine learning or data science. This makes the case for the ability to compare the performance of any set of algorithms in any application to find the best performing one. Currently however, analyzing algorithms is difficult for anyone without existing knowledge of the subject, regardless of the chosen method. In this thesis paper, we demonstrate the Algorithm Analysis program, a Julia package that provides a simple and more accessible way to analyze any optimization algorithm by leveraging the Lyapunov-based analysis approach. Simultaneously, we show that by creating a language specific to the domain of optimization algorithms, the Algorithm Analysis package can serve as a platform on which different methods to perform algorithm analysis can be implemented.
\end{abstract}

% the title page is the first numbered page (not the abstract),
% but the first page number appears on the table of contents
\setcounter{page}{3}

% table of contents
\tableofcontents

% list of figures
\lof

% dedication (optional)
\chapter{Dedication}

I would like to dedicate this thesis to my family and close friends.

% acknowledgements (optional)
\input{chapters/acknowledgements}

% acronyms (optional)
\chapter{Acronyms}

\begin{acronym}
  \acro{FG}[FG]{Fast Gradient}
  \acro{TCP/IP}[TCP/IP]{Transmission Control Protocol/Internet Protocol}
\end{acronym}

%%%%%%%%%%%%%%%%%%%%%%%%%%%%%%%%%%%%%%%%%%%%%%%%%%%%%%%%%%%%%%%%%%%%%%%%%%%%%%%
%% CHAPTERS
%%%%%%%%%%%%%%%%%%%%%%%%%%%%%%%%%%%%%%%%%%%%%%%%%%%%%%%%%%%%%%%%%%%%%%%%%%%%%%%
\mainmatter
\chapter{Introduction to JuPE}

Optimization problems can be in the broadest sense described as problems where an optimal solution is obtained using a limited amount of resources. Many problems that exist in the field of engineering and natural science can be categorized as optimization problems. For example, when mapping applications are used to navigate between two points, an algorithm tries to find the shortest path to a destination by choosing the direction of travel while under constraints such as traffic laws or avoiding road work.
Gradient-based iterative algorithms are a prominant tool to solve large optimization problems. Their ability to efficiently optimize functions without requiring an explicit formula means they are extensively used in fields such as machine learning and data science. As the term encompasses many algorithms, each can solve any of the many types of optimization problems with varying levels of speed and accuracy, making it very useful the ability to compare algorithms on specified metrics and identifying the one best suited for a problem. But while these algorithms are widely, their analysis have only been available to those with an in depth knowledge of the underlying math until recently \cite{pepit}. The main work of this thesis presents a tool for analysis of gradient-based algorithms' performance characteristics accessible to non-experts.

JuPE (Julia Performance Estimation) is a computer program written in the Julia programming language that automatically and systemically finds the worst-case performance guarantee of an algorithm at solving a specified set of problem. After the program is given a class of functions, the algorithm to be analyzed and the performance metrics, it returns a guarantee speed at which the algorithm inputted can solve any function in the provided set.


%%%%%%%%%%%%%%%%%%%%%%%%%%%%%%%%%%%%%%%%%%%%%%%%%%%%%%%%%%%%%%%%%%%%%%%%%%%%%%%%
\section{Optimization problems and algorithms}

In this paper, the optimization problem considered is in the form of finding the minimum point of a continuously differentiable function:
\begin{subequations}\label{opt prob}
  \begin{align}
    \textrm{minimize} \quad f(x) \\
    \textrm{subject to} \quad x \in X
  \end{align}
\end{subequations}
Where \(f(x)\) is the optimization function and \(X\) is a constraint set. Here, \(x\) is the input and \(f(x)\) is a measure of how close a solution is to being optimal. Well-known examples of this problem are large language models (LLMs) such as ChatGPT and machine learning models that enable self-driving features in automotives, amongst many others. These models are only possible due to the minimizing of loss functions, a fundamental part of their training where a function is used to quantify the dissimilarity between a model's output and the target values, and the model's parameters are modified iteratively in order to minimize the function and improve the model's performancec.

While traversing any function can give its minimum, for large-scale and complex problems, it is more efficient to be optimize numerically using iterative gradient-based  algorithms. These algorithms minimize a function by starting at an initial point \(x_{0}\) and iteratively updating \(x_k\) (\(k\) representing the current iteration number) using the gradient of the function at the last iteration $\nabla$ \(f(x_k)\) until it reaches a local minimum \(x_*\). For example, the gradient descent (GD) algorithm updates \(x_k\) following this formula:
\begin{equation}\label{eqn:GD}
  x_{k+1}=x_{k}-\alpha \nabla f(x_k)
\end{equation}
Where $\alpha$ is the step size, an adjustable parameter of the algorithm. $\alpha$ can affect the speed at which the algorithm converges, or whether it converges at all, in which case overshooting occurs. Following this update formula, in each iteration, \(x\) moves toward the goal \(x_*\). Accelerated algorithms exist that seek to solve the problem of overshooting, such as Polyak’s Heavy Ball (HB) method which introduces a momentum that incorporates previous iterations of \(x\):
\begin{equation}\label{eqn:HB}
  x_{k+1}=x_k-\alpha \nabla f(x_k)+ \beta (x_k-x_{k-1})
\end{equation}
While Nesterov’s Fast Gradient (FG) evaluates the gradient at an interpolated point:
\begin{subequations} \label{eqn:FG}
  \begin{align}
    x_{k+1}     &=x_k-\alpha \nabla f(y_k), \label{eq_state}       \\
    y_{k+1} &=x_{k+1}+\beta (x_{k+1}-x_k) \label{eq_interpolated point}
  \end{align}
  \end{subequations}
%%%%%%%%%%%%%%%%%%%%%%%%%%%%%%%%%%%%%%%%%%%%%%%%%%%%%%%%%%%%%%%%%%%%%%%%%%%%%%%%
\section{Algorithm analysis}
Let us consider the problem of minimizing a simple quadratic function:
\begin{equation} \label{eqn:quadratic}
    f(x) = x^2/2 - 3*x + 4
  \end{equation}
Using (GD) and equation \ref{eqn:GD}, substituting step size $\alpha$ with values 0.2, 0.5, and 2, and picking a starting point of $x_0$. By solving the problem with each variant of the algorithm and counting the number of iterations each runs for before reaching within 0.001 of the true minimum, we can measure the iteration complexity:

It can be seen how different tunings on the same algorithm can achieve drastically different speed and accuracy optimizing a function, or whether it can solve for the minimum at all. Considering there exist many other first order methods in addition to the three in 1.1, each infinitely adjustable by changing the step size or by changing the number of past iterations used, being able to predict how an algorithm will perform before solving an optimization problem can mean a more accurate solution can be found and in fewer iterations. And while the example is of a simple function where overshoot in the case of the third tuning can easily be identified, and the number of iteration needed is small, the benefit of using an optimal algorithm only increases as the problem gets bigger and more complex. Using the same example of training large language and self-driving models, the training process has been and might need to go on for years as more training data is available and the models need to continously improve all while using vast amounts of computational power. As a result, even a small improvement in the performance of the algorithm used can lead to large savings in time and energy.

Considering the quadratic function example, while it yielded an analysis of the algorithms' performance, it required solving the optimization problem. Not only would this be computationally costly for any problem large enough to warrant being optimized numerically in the first place, any benefit finding an algorithm better at solving that problem is negated as the problem has already been solved. Additionally, any analysis result is applicable only to one function and cannot be reliably used to derive a first-order method's performance on any other problem.

Due to these limitations, it is more efficient to analyze algorithms' performance at solving a broader set of problem. As a result of their widespread application, popular iterative gradient-based algorithms have been extensively analyzed, a frequently cited example being \cite{adam}. In this paper, the author designed and conducted experiments where how the algorithms perform in typical applications is recorded. While this approach is can provide a general evaluation of an algorithm's performance, it is done emperically. Instead, there exists approaches toward analyzing algorithms \cite{drori2012}, \cite{taylor2016}, and \cite{lessard2016} that aim to find a mathematically provable performance guarantee of an algorithm over a class of functions. This worst-case analysis is referred to as algorithm analysis: Given that a characteristic that a set of functions might share (such as being convex or quadratic), algorithm analysis would return the worst-case performance measure that guarantee the algorithm analyzed would perform as good or better solving every function within said set.

\section{Julia programming language}

JuPE is written in the Julia programming language, a high-level programming language designed specifically for high-performance numerical computing. Julia's compiler performance has been benchmarked to be faster than many other languages used for numerical computing while rivalling C, a language often used for its high efficiency \cite{julia}. Julia accomplishes this while being a high-level language with simple syntax rules that resembles existing popular languages, making it easy to code with and to understand.

Julia was also chosen as it is designed for numerical computing, supporting matrices as well as UTF-8 encoding, making it possible to use scientific notation: variables and functions as they exist in the code and as the user inputs them into the program can use math symbols or Greek letters. This makes Julia excel at communicating mathematical concepts, which simplifies both the process of coding the program and understanding its mathematical underpinnings.

Julia was also chosen as it is open-source and available for free. As JuPE is a package designed for expert and novice users alike to install and use, it made sense to choose Julia as it available on many of the popular platforms such as macOS, Windows, and Linux.

\section{Overview}

JuPE performs worst-case algorithm analysis when three main inputs are provided: The class of functions in question, the algorithm being analyzed, and a performance measure. The package then performs the algorithm analysis and returns the fastest guaranteed convergence rate.

Users can pick from one of the algorithms provided in the package or create their own iterative first-order algorithm by specifying how it is updated. The class of functions can be provided by detailing the characteristic of the set, such as 1 strong 10 smooth convex function. Users can specify a performance measure, which can be how far the iterate \(x_k\) is from the goal \(x_*\) or any quadratic combinations of the iterates. Throughout the process, the user never has to change the code of the package or understand how JuPE works, making it an easy to use black box tool.

In the next chapters, we will 1) discuss the mathematical approach that JuPE utilize, 2) break down the code structure of the program and how it functions, and 3) show some of the analysis that the package has done.
 \chapter{Literature Review}

In recent years, numerous studies have been conducted comparing the performance of optimization algorithms, particularly in machine learning and deep learning contexts. These comparisons often focus on convergence speed, accuracy, and robustness across various models and datasets. In \cite{adam}, Kingma and Ba presented the Adam (Adaptive Moment Estimation) algorithm. In order to demonstrate the Adam algorithm, the authors designed and conducted experiments where different optimization algorithms are used to solve popular machine learning models, the convergence rate of each recorded. The result of these experiments is emperical evidence providing a general idea on an algorithm's performance at solving different problems.

There exist in the literature many approaches to approaches to performing algorithm analysis. In 2014, Drori and Teboulle first introduced the method of representing a class of function with constraints, reformulating the problem of analyzing an optimization method into a semidefinite program (SDP) whose size is proportionate with the number of iterations the algorithm is run.\cite{drori2012}. The paper coined the term Performance Estimation Problem (PEP) and showed that by solving convex semidefinite problem, a worst-case numerical bound on an algorithm's performance solving that class of function can be derived. Taylor, Hendrickx and Glineur built upon this work by introducing the ideas of creating a finite representation for a class of smooth strongly convex functions using closed-form necessary and sufficient conditions. While the above mentioned two approaches give the performance bound in the form of a guarantee how close \(x_k\) is to the goal \(x_s\) after a fixed number of iterates, the method used in this package proves that the performance measure inputted by the users decreases at a guaranteed rate throught out the optimizing process.

In \cite{iqc}, Megretski and Rantzer demonstrated how integral quadratic constraints (IQCs) can be used to unify and simplify the analysis of system stability and performance. The paper shows how a complex system can be described using certain IQCs and presents a stability theorem for systems described by IQCs.

Computer programs have been developed to perform algorithm analysis using the PEP methods, which are PESTO \cite{pesto}, a MATLAB toolbox, and PEPit \cite{pepit}, a Python package. The program is able to perform algorithm analysis and generate a worst-case performance guarantee for algorithms and function classes from a supported list. PESTO and PEPit follows the PEP methodology and first presented an automatic way to analyze gradient-based algorithms.

The main contribution of this thesis paper is to create an computer program similar PESTO and PEPit that aims to provide an accessible and fast way to analyze the performance of first-order methods for a guaranteed convergence rate, leveraging the approach presented in \cite{tutorial}, in the Julia programming language.
\chapter{Lyapunov-based approach}

JuPE performs algorithm analysis by using the technique layed out by Van Scoy and Lessard in \cite{tutorial}. While JuPE is a blackbox tool, understanding the mathematical approach on which it is based is prequisite to understanding the package's code and functionalities.

The steps of this technique, which will be discussed in detail over the sections of this chapter, consists of 1) viewing the algorithm as a Lur'e problem, 2) Replacing the nonlinear gradient with interpolation conditions that represent the class of smooth strongly convex functions, 3) Use lifting matrices to tighten to the interpolation condition representations, and 4) Prove whether a convergence rate is guaranteed by solving a convex semidefinte program.
%%%%%%%%%%%%%%%%%%%%%%%%%%%%%%%%%%%%%%%%%%%%%%%%%%%%%%%%%%%%%%%%%%%%%%%%%%%%%%%%
\section{Iterative algorithms as Lur'e problems}

The first step in the technique is to view optimization algorithms from a control theory perspective: As iterative gradient-based algorithms uses the gradient of the function to update \(x\), they can be reformulated into a linear time-invariant (LTI) system (how the algorithm update) in feedback with a static nonlinearity (the gradient of \(f\)) at point \(x\). Using a block diagram, the algorithm can be seen as:

Here, \(G\) represents the LTI system, while \(y\) and \(u\) are input and output of the gradient nonlinearity. For example, (FG) equation \ref{eqn:FG} can be rewritten to match this view as:
\begin{subequations} \label{eqn:FG2}
	\begin{align}
	  x_{k+1}     &=x_k-\alpha u_k \label{eq_FGstate},       \\
	  y_{k+1} &=x_{k+1}+\beta (x_{k+1}-x_k), \label{eq_FGinterpolated point}, \\
	  u_k &= \nabla f(y_k) \label{eq_FGggradient}
	\end{align}
	\end{subequations}
The algorithm can then be put into state-space representation as:
\begin{subequations} \label{eqn:FGss }
	\begin{align}
	  \xi_{k+1} &= \bmat{(1+\beta) & -\beta\\ 1 & 0}\xi_k  + \bmat{-\alpha\\ 0}u_k \label{eq_FGssstate}, \\
	  y_k &= \bmat{1+\beta & \beta}\xi_k, \label{eq_FGssinterpolation},\\
	  u_k &= \nabla f(y_k) \label{eq_FGssgradient}
	\end{align}
	\end{subequations}

The LTI system \(G\) can be expressed with four matrices that change in value depending on the algorithm and size depending on the number of past states used to update \(x\). For (GD), (FG), and (HB) as they are described in 1.1, these matrices are:

Beyond the three listed examples, this step can be applied to any other first order methods. Deriving an LTI system represented by matrices out of an algorithm not only creates a representation easy to understand and operate on, but also in next sections enable the representation of the past states of an algorithm and the forming of a Lyapunov function that will guarantee a convergence rate.

\section{Interpolation condition}

For the remainder of this paper, we will focus on the only class of function currently supported by JuPE, m strong L smooth convex function \(f \in \) \(F_{m,L}\).

While an LTI system is relatively simple to solve, the addition of the nonlinearity representing the gradient of the function is not. This step replaces the gradient of a smooth strongly convex function with a set of conditions on the input and output of that linearity using the characteristics of the class of function. This step relies on a theorem first presented in \cite{taylor2016} and reformatted in \cite{tutorial}:

\begin{theorem}[\cite{taylor2016}, Thm. 4; \cite{tutorial}, Thm. 3]
	\label{thm:interpolation_condition}
	Given index set \(I\), a set of triplets \({(y_k, u_k, f_k)}_{k \in I}\) is \(F_{m,L}-interpolable\), meaning there exists a function \(f \in F_{m,L}\) satisfying \(f(y_k) = f_k\) and \(\nabla f(y_k) = u_k\) if and only if:

	\(2(L-m)(f_i - f_j) - mL||y_i - y_j||^2 + 2(y_i - y_j)^{T}(mu_i - Lu_j) - ||u_i - u_j||^2 > 0\) for \(i ,j \in I\)
\end{theorem}
Where the Euclidean-norm of a vector \(x\) is denoted as \(||x||\). Using this inequality, constraint matrices \(M\) and \(m\) can be constructed:

When the interpolation conditions are satisfied, \(M\) and \(m\) are positive definite, meaning they are symmetric and and its eigenvalues are positive.

This step in the methodology is similar to that in \cite{pepit}. By using interpolation conditions, the smooth strongly convex class of function inputted is represented by a matrix that is positive semidefinite.

\section{Lifting algorithm}

\section{Lyapunov function certification}


\chapter{Code Components}

To in order to perform algorithm analysis with JuPE, the user need to follow the folliowing 3 steps:
\begin{enumerate}
	\item Choose from a supported list the class of function to be optimized.
	\item Define an algorithm to be analyzed .
	\item Specify a performance measure.
  \end{enumerate}
 
\begin{figure}[hbtp]
    \caption{Analysis Example}
    \label{ex_analysis}
	\begin{lstlisting}[mathescape]
	m,L = 1,10
	$ \alpha $ = 2/(L+m)
	@algorithm begin

		f = DifferentiableFunctional{R$ ^n $}()
		xs = first_order_stationary_point(f)
		f' $ \in $ SectorBounded(m, L, xs, f'(xs))

		x0 = R$ ^n $()
		x1 = x0 - $ \alpha $*f'(x0)

		x0 => x1

		performance = (x1-xs)^2
	end

	@show rate(performance)
\end{lstlisting}
\end{figure}

In the example code, the user while using JuPE's provided macro to simplify the input process:
\begin{itemize}
	\item Defined the class of function f and its gradient f' by calling one of the provided functions, in this case 1 smooth 10 strong convex functions.
	\item Set the global minimum goal as a stationary point $ x_s $
	\item Defined an initial state $ x_0 $ and specified the algorithm with which the state is updated using algebra - gradient descent with a step size 2/11 in this example.
	\item Set the performance measure as the norm distance between the updated point and the goal $ (x_1 - x_s)^2 $ .
\end{itemize}

With the calling of the "rate" function, JuPE derives every necessary input from the performance measure and performs analysis automatically to return a rate of 0.6687164306640625, which can be seen in Figure~\ref{ex_result}. This means for the provided algorithm and every function in the class, the convergence rate of the performance measure is guaranteed to match or exceed the result worst-case guarantee rate.
 
\begin{figure}[hbtp]
    \caption{Analysis result}
    \label{ex_result}
\begin{lstlisting}
rate(performance) = 0.6687164306640625
\end{lstlisting}
\end{figure}

JuPE performs algorithm analysis by following the set of instructions presented in section 3 to create and solve an optimization problem and derive a performance certification. Implenmenting a mathematical procedure as code presents a list of challenges which includes being able to understand and differentiate between variables, represent concepts such as gradients or states, or formulating and solving a convex optimization problem, while keeping the users' interaction with the program simple. This chapter goes into the code that constitutes JuPE and enables these functionalities.

%%%%%%%%%%%%%%%%%%%%%%%%%%%%%%%%%%%%%%%%%%%%%%%%%%%%%%%%%%%%%%%%%%%%%%%%%%%%%%%%
\section{Expressions}

Expressions are data structures acting as the smallest building block upon which every other concepts are built. An expression data structure contains the following fields:
\begin{description}
	\item[Label] When an expression is defined, the variable name used is stored in the expression as a string
	\item[Value] Stores the value or decomposition of an expression.
	\item[Constraints] When a constraint is applied to an expression, such as an expression being positive or negative semidefinite, the constraint is stored in the 'constraints' field.
	\item[Oracles] When an expression is defined by sampling one or multiple oracles, the oracles used are stored in the expression data structure
	\item[Next] When a state is defined as the 'next' state of another state, it is stored in the original state under the 'next' field.
 \end{description}

\subsection*{Value and Decomposition}
Each expression contains in its 'value' field either a scalar value, a vector value, or a decomposition dictionary. Depending on the content of the 'value' field, expressions can be categorized as:
\begin{description}
	\item[Variable expressions] an expression defined to be in a field. Its value is empty and its decomposition is itself.
\begin{figure}[hbtp]
	\caption{Example of a variable expression}
	\label{ex_variable}
		\begin{lstlisting}[mathescape]
	x0 = R$^n$()
	@show(x0)
	x0 = Variable{R$^n$}

	Vector in R$^n$
		Label: Variable{R$^n$}
		Oracles: LinearFunctional{R$^n$}
		Associations: Dual => LinearFunctional{R$^n$}	
		\end{lstlisting}
\end{figure}
	\item[Decomposition expressions] formed by performing algebra on other expressions. Its value is a decompostion data structures containing how many of each variable expressions forms the decomposition expression.

\begin{figure}[hbtp]
	\caption{Example of a decomposition expression}
	\label{ex_decomposition}
		\begin{lstlisting}[mathescape]
	decomposition_exp = (x0-xs)$^2$
	@show decomposition_exp 
	decomposition_exp = -<x0,xs> + |x0|$^2$ - <xs,x0> + |xs|$^2$

	Scalar in R
		Decomposition: -<x0,xs> + |x0|$ ^2 $ - <xs,x0> + |xs|$ ^2 $
		\end{lstlisting}
\end{figure}

In the 
\end{description}

\subsection*{Fields}
In example \ref{ex_variable}, expression 'x0' is defined to be in field R$^n$, a field pre-defined in JuPE which encompasses n-dimensional.

\subsection*{Algebra}
Expressions in JuPE belong in an inner product space, which is a set of elements that can be vectors or numbers. Inner product spaces allow a list of operations, all of which are supported by JuPE, including:
\begin{description}
	\item[Addition or subtraction between expressions] Vectors and numbers can be added together in an inner product space. In an addition operation, if both expressions of the operation posess a 'value', they are added to create the value of a new resulting expression. Otherwise, the result is a new expression is created whose decomposition is the merging of the expressions being combined's decompositions.
	\item[Multiplication or division between an expression and a scalar] This operation scales the value or decomposition of an expression by the scalar value.
	\begin{figure}[hbtp]
		\caption{Example of addition, subtraction and scalar operation}
		\label{ex_algebra1}
		\begin{lstlisting}[mathescape]
	@algorithm begin
		x0 = R$^n$()
		y0 = R$^n$()
		a = x0-2*y0
		b = a+2*x0
	end
	@show(a)
	a = a

	Vector in R$^n$
		Label: a
		Decomposition: -2 y0 + x0
		Associations: Dual => a*
	@show(b)
	b = b
	
	Vector in R$^n$
		Label: b
		Decomposition: -2 y0 + 3 x0
		Associations: Dual => b*
	
		\end{lstlisting}
	\end{figure}

	\item[Inner product operation between two vectors] In an inner product space, the inner product of two vectors result in a scalar.
	\item[Squared norm] An expression of the normed vector vpace type can be squared to produce a an inner product space expression.
	\begin{figure}[hbtp]
		\caption{Example of an inner product between two vectors and norm of vector}
		\label{ex_innerandnorm}
		\begin{lstlisting}[mathescape]
	@algorithm begin
		inner = x0'*y0
		norm = x0^2
	end
	@show(inner)
	inner = <y0,x0>

	Scalar in R
		Label: <y0,x0>
		Oracles: x0*
	@show(norm)
	norm = |x0|$^2$
	Scalar in R
		Label: |x0|$^2$
		Oracles: x0*
		\end{lstlisting}
	\end{figure}
\end{description}

\section{Oracles}
As JuPE perform analysis over sets of functions using only constraints on the \( \nabla (f) \) block of, the block can be treated as a blackbox. In JuPE, these blackboxes are represented by oracles, data strucutres containing the relation and constraint information between expressions. Each oracle represent a class of function and can only exist if there exist interpolation conditions for said class. Oracles can be sampled at an expression to return another expression, establishing the relation information between the two expressions. As an oracle is sampled, JuPE uses the set of interpolation conditions to create every constraints on the two expressions.  In \ref{ex_analysis}, by calling the SectorBounded function, an oracle containing the interpolation conditions for 1 strong 10 smooth convex functions is created and labeled f', and the oracle is sampled at points xs and x0 by defining f'(xs) and f'(x0) inside the labeling macro.

\subsection*{Inner product oracle}
Different from other algebraic operations supported by JuPE, the inner product is derived by creating an oracle of one vector and sampling it at the other vector of an operation, returning a new variable expression. As both the transpose of a vector, which is used to calculate the inner product, and the gradient of a function uses the notation " ' ", oracles are used to assist in the formation of inner product expressions. In this case, the oracle is not based on any class of function and therefore attribute no constraint to the vector being sampled.

\section{States}
JuPE represents the states of an algorithm using expressions. As the user inputs the algorithm being analyzed, an initial state is created and an updated state is defined as some algebraic combination of the initial state and the gradient. The relationship between a state and its next state can be defined using the "=>" operation inside the labeling macro, as can be seen in \ref{ex_analysis}, and the next state is stored in the "next" field of a state expression.

%Variable expressions that make up the next state must be in the same field as those of the previous state.

\section{Label}
JuPE uses macro to keep the process of providing inputs to the program simple. As some of the programming rules of programming might be difficult to navigate for users who might not be used to programming, in order to make the process of using JuPE as accessible as possible, JuPE's macro:
\begin{description}
	\item[Describe] When an expression is referenced, JuPE will describe the expression and any relevant field without the user having to access it.
	\item[Define] During the inputing process, users can define a new expression or oracle with only one line specifying its trait, instead of the usual steps of declaring a new object and filling in its fields that typically exist in programming. The macro will calls the necessary functions to create the object, assign every relevant field as well as updating every  object associated with the one being created.
\end{description}

\section{Constraints}
As part of the analysis process, JuPE uses the interpolation conditions of the class of function inputted by the user to create constraints similar to section 3.2. To support constraints, JuPE uses data structures that include the expression constrained and one of the supported sets of values defined by the constraints.

When an oracle is sampled, a set of constraints on the expression being sampled and the result of the sample - the input and output of the black box - is created, and constraints are added as the oracle is sampled at more than one expression. For each constraint, the constraint is added to each variable expressions associated with it. The figure below shows the constraints created by the oracle in \ref{ex_analysis}:

\begin{figure}[hbtp]
	\caption{Example of oracle created constraint}
	\label{ex_orc_constraints}
	\begin{lstlisting}[mathescape]
vars, cons, orcs = variables_constraints_oracles(performance)
cons
	Set of constraints with 3 elements:
		0 $<=$ 1.1 <$\nabla $f(x0),x0> + 2.0 <xs,x0> - 2.0 |x0|$^2$ + 1.1 <x0,$\nabla $f(x0)> - 2.0 |xs|$^2$ - 1.1 <xs,$\nabla $f(x0)> + 2.0 <x0,xs> - 0.2 |$\nabla $f(x0)|$^2$ - 1.1 <$\nabla $f(x0),xs>
		0 $<=$ R[|x0|$^2$ <xs,x0>; <x0,xs> |xs|$^2$]
		0 $<=$ R[|x0|$^2$ <$\nabla $f(x0),x0> <xs,x0>; <x0,$\nabla $f(x0)> |$\nabla $f(x0)|$^2$ <xs,$\nabla $f(x0)>; <x0,xs> <$\nabla $f(x0),xs> |xs|$^2$]
	\end{lstlisting}
\end{figure}

In addition to being created by sampling oracles, constraints can also be defined by users. When analyzing any algorithm, constraints can be added to the initial condition of the algorithm. Any constraints added by the user is included in the formation of the optimization problem used to derive the performance bound.

\begin{figure}[hbtp]
	\caption{Example of user added constraint}
	\label{ex_user_constraints}
	\begin{lstlisting}[mathescape]
	@algorithm begin
		(x0-xs)^2 $<=$ 1
	end
	\end{lstlisting}
\end{figure}
\section{Performance measure}
Part of JuPE's required inputs is the performance measure, the convergence rate of which JuPE finds the worst-case guarantee through algorithm analysis. In \ref{ex_analysis}, the performance measure is set as $ (x_0 - x_s) ^2 $, which is the norm or distance between the initial point and the goal, which means the convergence rate guarantee returned is that of the distance between the point updated using gradient descent after each iteration $ x_k $ and the goal $ x_s $.
\section{JuMP}
 
\section{Lyapunov function formulation}
\chapter{Analysis Process and Result}\label{chapter:result}

The Lyapunov function approach require 2 components which forms the semidefinite problem described in \Cref{chapter:lyapunov} to produce a worst-case performance convergence rate: the state update matrices and the constraints formed by the interpolation conditions. These components are derived from the input provided to the program which undergo transformations before they can be used to create an optimization problem in the JuMP modelling language, the process of which can be described in 3 steps:
\begin{enumerate}
    \item The Algorithm Analysis program automatically uses the input provided to form a systematic charaterization the analysis problem using data structures described in \Cref{chapter:code}, including how the algorithm being analyzed updates, the constraints created by the interpolation conditions of the class of function and the performance measure.
    \item These data structures are converted to real number vectors and matrices that represent the updated state of the algorithm and each algorithm in the form of the a linear function of the initial states and inputs.
    \item An optimization problem is created inside a JuMP model using these representations and solved to verify whether a certain convergence rate is feasible for a given problem. This process is repeated with different convergence rates as the program search for the lowest feasible convergence rate.
\end{enumerate}

This chapter details the analysis process of analyzing gradient descent's performance at optimizing any $1$-$10$ sector bounded function as shown in Figure~\ref{ex_analysis}, including how these steps presented in \Cref{chapter:lyapunov} are performed and how the optmization problem is formed and solved to derive worst-case performancce convergence rate.

\subsection*{Real scalars and linear form}

% In JuPE, when variable expressions are defined in an inner product space, they are vector. These expressions include states of an algorithm, the starting and end points, and the gradient of a function at a point. This is because a function class is multidimensional, which means each point or iterate is a vector whose size is identical to the dimension of the function. 

% All three components needed to form the linear matrix inequality - the Lyapunov functions $V(x_k)$ and $V(x_{k+1})$ and the left hand side of each constraint - from equations \ref{eqn:int_cond2}, \ref{eqn:Ly_ineq} and \ref{eqn:Ly_ineq2} are functions linear in the elements of the Gram matrix and optimization variables. On the other hand, the JuMP modeling language does not support the expression and constraint data structures presented in chapter 4, and the LMIs of \ref{Ly_ineq2} must be transformed into a function linear in the optimization variables and real numbers before it can be formed inside the JuMP model. 
As specified in \ref{Lyapunov}, the linear matrix inequalities are constructed from the linear form of the Lyapunov functions and the constraints as a function of the vector $[x; u]$. This process is done in three steps, which are:
\begin{enumerate}
    \item Of every expressions that has been created during the input process, define the initial state vector \texttt{x} as every real expressions which contain another expression in its next field.
\begin{figure}[h!]
    \begin{lstlisting}[mathescape]
x  = collect(v for v $\in $ vars if !ismissing(next(v)) && v isa R)
4-element Vector{R}:
    |x0|$^2$
    |xs|$^2$
    <x0,xs>
    <xs,x0>
\end{lstlisting}
\caption{Initial state real scalar expressions from example \ref{ex_analysis}}
\label{ex_initstate}
\end{figure}

    \item Define an update state vector x$^+$ consisting of the next expression of every expression in the initial state. The input vector u is then defined as every real expression that exist in the decomposition of the updated state expressions but not in the initial state expressions.
    \begin{figure}[h!]
        \begin{lstlisting}[mathescape]    
x$^+$ = next(x)
4-element Vector{R}:
    <x0,xs> - 0.18181818181818182 <$\nabla $f(x0),xs>
    -0.18181818181818182 <$\nabla $f(x0),x0> + 0.03305785123966942 |$\nabla $f(x0)|$^2$ - 0.18181818181818182 <x0,$\nabla $f(x0)> + |x0|$^2$
    -0.18181818181818182 <xs,$\nabla $f(x0)> + <xs,x0>
    |xs|$^2$

u  = collect(setdiff(variables(x$^+$), variables(x)))
5-element Vector{Expression}:
    <xs,$\nabla $f(x0)>
    <x0,$\nabla $f(x0)>
    <$\nabla $f(x0),xs>
    <$\nabla $f(x0),x0>
    |$\nabla $f(x0)|$^2$
        \end{lstlisting}    
    \caption{Updated state and input real scalar expressions from example \ref{ex_analysis}}
    \label{ex_updatedstate_input}
    \end{figure}
    \item The initial state and input vector \texttt{[x; u]} is the code equivalent of $\bmat{x & u}^\tp$ and can form every expressions required to form the linear matrix inequality. This transformation, which will be refered to as the linear form of an expression, can be derived by finding the values of each expression in the vector \texttt{[x; u]} present in the expression's decomposition dictionary.
\end{enumerate}

\begin{figure}[h!]
    \begin{lstlisting}[mathescape] 
linear_form = vec(linearform([x; u] => x0^2 - 3*(x0'*xs)))
linear_form'*[x; u]
Scalar in R
  Decomposition: -3 <xs,x0> + |x0|$^2$
\end{lstlisting}    
\caption{Example of linear form of a scalar expression \ref{ex_analysis}}
\label{ex_linearform}
\end{figure}

\section{Performance measure}
The linear form matrix of the performance measure is the first of the three components needed to form the Lyapunov function ($\|x_k - x_s\|^2$ in \eqref{eqn:Ly_ineq}). For example, the performance measure in \ref{ex_analysis}, which is defined as $(x0-xs)^2$ and which evaluates into $|x0|^2 - <xs, x0> - <x0, xs> + |xs|^2$, has the linear form presented in Figure~\ref{ex_linearform2}.
\begin{figure}[h!]
\begin{lstlisting}[mathescape]
$\mathcal{P} $ = vec(linearform( [x; u] => performance ))
print($\mathcal{P}$)
[-1, -1, 1, 1, 0, 0, 0, 0, 0]
\end{lstlisting}
\caption{Linear form matrix of expression $(x0-xs)^2$}
\label{ex_linearform2}
\end{figure}

\section{Algorithm, state update and Lyapunov function formulation}
% \subsection*{State space matrix and their linear form formulation}
As shown in Figure~\ref{ex_analysis} and in section \ref{states}, the algorithm to be analyzed is provided as input first by defining an initial state and how the next state is updated from the initial state. The initial state is defined to be a vector in an inner product space, and the updated state is a linear function of one or multiple initial state and the gradient of the function evaluated at some point. While the gradient descent algorithm updates using only one state and evaluate the gradient at the previous state, if an algorithm updates using multiple past states or the gradient at an interpolated point, these vectors will also have to be defined. The forming of the algorithm can then be completed by defining the relationship between states and their next states using the \texttt{=>} operation, which updates the next field of every expression in the decomposition of which there is the state on the left hand side of the operation. For example, in Figure~\ref{ex_analysis}, the state vector \texttt{x1} is defined as a function of the state vector \texttt{x0} and as the next state of \texttt{x0}, while the next state of the stationary point \texttt{xs} is itself. This not only means the \texttt{next} field of \texttt{x0} and \texttt{xs} are \texttt{x1} and \texttt{xs} respectively, but also next state of any every expression derived from the norm, inner product, or algebraic calculation of which \texttt{x0} is a part is that calculation done with \texttt{x1} instead. In this example, as shown in Figure~\ref{ex_next}, the next state of the inner product of $x_0$ and $x_s$ denoted as \texttt{next(x0'*xs)} is the inner product of $x_1$ and $x_s$ \texttt{x1'*xs}. This enables the operation in Figure~\ref{ex_linearform_next_state} and allow any updated iterate to be automatically expressed a linear function of the intial states and inputs.
\begin{figure}[h!]
	\begin{lstlisting}[mathescape]
next(x0)

Vector in R$^n$
  Label: x1
  Decomposition: -0.18181818181818182 $\nabla $f(x0) + x0
  Associations: Dual => x1*

next(x0'*xs)

  Scalar in R
    Decomposition: -0.18181818181818182 <xs,$ \nabla $f(x0)> + <xs,x0>
\end{lstlisting}
\caption{\texttt{next} field of state a vector expression and a scalar formed from a state expression}
\label{ex_next}
\end{figure}

As the initial state vector is defined in Figure~\ref{ex_initstate} and the updated state vectors in Figure~\ref{ex_updatedstate_input}, their linear form matrices is the second of the three components needed to formulate the Lyapunov function and can be formed as shown in Figure~\ref{ex_linearform_state} and Figure~\ref{ex_linearform_next_state}.

\begin{figure}[h!]
    \begin{lstlisting}[mathescape]
X  = linearform([x; u] => x)
    4x9 Matrix{Int64}:
    1  0  0  0  0  0  0  0  0
    0  1  0  0  0  0  0  0  0
    0  0  1  0  0  0  0  0  0
    0  0  0  1  0  0  0  0  0    
    \end{lstlisting}
    \caption{Linear form state matrix x}
    \label{ex_linearform_state}
\end{figure}
\begin{figure}[h!]
    \begin{lstlisting}[mathescape]
X$^+$ = linearform([x; u] => x$^+$)
4x9 Matrix{Real}:
1  0  0  0   0        0       -0.1818   0         0
0  1  0  0  -0.1818   0        0        0.03306  -0.1818
0  0  1  0   0       -0.1818   0        0         0
0  0  0  1   0        0        0        0         0
\end{lstlisting}
\caption{Linear form state matrix x$^+$}
\label{ex_linearform_next_state}
\end{figure}
% \subsection*{State space in Lyapunov function}
Following the steps presented in chapter 3, the Lyapunov function can begin to be formed by first defining an optimization variable $P$ in the JuMP model as a JuMP variable. Once JuMP and the solver start optimizing the problem, P is one of the variable that will be optimized to produce a solution. The Lyapunov functions are then created following \eqref{eqn:Ly_ineq} but with code variables as:

\begin{subequations} \label{eqn:Ly_ineq3}
	\begin{align}
	    L1 = \mathcal{P} - X^\tp P     \\
        L2 = X^{+\tp} P - \rho X^\tp P
	\end{align}
\end{subequations}

\section{Constraints}
As presented in \ref{oracles}, the oracle created from the class of function and the transpose of each expression automatically forms the interpolation condition and Gram matrix constraints. These constraints are linearized and added to the optimization in 2 steps:
\begin{description}
    \item [Optimization variable multipliers] For each constraint $i \in$, two optimization variables $\lambda_i$ and $\mu_i$, which represent $\Lambda^1_i$ and $\Lambda^2_i$ in \Cref{chapter:lyapunov} are defined as JuMP variables. If the constraint is applied to the Gram matrix, the optimization variable will be a matrix sharing the same size with the matrix constrained. Otherwise, if the constraint is created from the interpolation conditions of the class of function and is applied to a single real scalar expression, the optimization problem created will have a size of 1.
    \item [Constraint on multiplier] The JuMP variables multipliers are constrained in the JuMP model depending on the constraint expression they were created for: The multiplier is not constrained if the expression is constrained to be zero, constrained to be non-negative if the expression is constrained to be non-negative, and constrained to be symmetrical and in the JuMP supported positive semidefinite cone if the expression is constrained to be positive semidefinite.
    \item [Linear form of constraints] The linear form of each constraint scaled by the multiplier is created and added to the Lyapunov functions.
\end{description}

If the expression constrained is a single real scalar, the linear form of the constraint is derived similarly to the linear form of the performance measure or state space matrices but scaled by the multiplier. Suppose we have the constraint $(x0 - xs)^2 \geq 0$ and matrix $\bmat{x\\ u} = \bmat{|x0|^2\\ <xs, x0>\\ |xs|^2}$, the linear form of the constraint in terms of $\bmat{x\\ u}$, denoted as $M$ would be:

\begin{subequations} \label{eqn:constraint_single}
	\begin{align}
    \lambda * (x0 - xs)^2 &= M * \bmat{|x0|^2\\ <xs, x0>\\ |xs|^2} \label{eq_cons_single1}       \\
	M &= \bmat{\lambda& 2\lambda & \lambda} \label{eq_cons_single2}
	\end{align}
\end{subequations}

If the expression constrained and its corresponding multiplier are vectors of expression, the linear form of the constraint is derived as the linear form of the inner product between the multiplier vector and the constraint expression vector. Suppose we have a constraint vector $\bmat{(x0-xs)^2 \\ (x0-xs)^2-3*|xs|^2} \geq 0$ and the same $\bmat{x\\ u}$ matrix as \eqref{eqn:constraint_single}, the linear form of the constraint in terms of $\bmat{x\\ u}$, denoted as $M$ would be:

\begin{subequations} \label{eqn:constraint_vector}
	\begin{align}
    \bmat{\lambda  & \lambda } * \bmat{(x0 - xs)^2 \\ (x0 - xs)^2-3*|xs|^2} &= M * \bmat{|x0|^2\\ <xs, x0>\\ |xs|^2} \label{eq_cons_vector1}       \\
	M &= \bmat{\lambda & -\lambda & \lambda \\ \lambda & -\lambda & -2\lambda} \label{eq_cons_vector2}
	\end{align}
\end{subequations}

And if the expression constrained and its corresponding multiplier are matrices, the linear form of the constraint is the linear form of the trace of the matrix multiplication between the multiplier and the constraint expression. For the Gram matrix in \eqref{eqn:trans_cond} which is constrained to be positive semidefinite, its linear form would be:

\begin{equation} \label{eqn:trans_cond}
	tr(\lambda \bmat{||x0||^2 & \innerproduct{xs}{x0} & \innerproduct{\nabla f(x0)}{x0} \\ \innerproduct{x0}{xs} & ||xs||^2 & \innerproduct{\nabla f(x0)}{xs} \\ \innerproduct{x0}{\nabla f(x0)} & \innerproduct{xs}{\nabla f(x0)} & ||\nabla f(x0)||^2]}) \geq 0	
\end{equation}

Where $\lambda $ is a 3x3 JuMP variable. In all three cases, for each constraints, 2 identical linear form matrices are created, one scaled by \texttt{$\lambda$} and added to the first Lyapunov function and the other by \texttt{$\mu$} and added to the second Lyapunov function. This completes the final linear matrix inequalities as defined in \eqref{eqn:LMI1}.

\section{Derived feasibility and bisection search}
Upon the completion of the linear matrix inequalities, the solver of the JuMP model is called to optimize the problem and find the variables \texttt{$P$}, \texttt{$\lambda$}s and \texttt{$\mu$}s for which the linear matrix inequality is satisfied and a convergence rate $\rho$ can be guaranteed.

Using the definition of the convergence rate in \eqref{eqn:convergence_rate}, a \texttt{$\rho$} value of 1 means the algorithm cannot be guaranteed to converge and a convergence rate of 0 means the algorithm is guaranteed to converge after a single iterate. In order to find the worst-case performance rate, the program performs bisection search, also known as binary search, for the smallest value $\rho$ between 0 and 1 that makes the optimization problem feasible, calling a function to perform the steps presented in this chapter for each value \texttt{$\rho$} and checking feasibility at each iterate of the search. The smallest value \texttt{$\rho$} found within a tolerance of $1E-5$ is returned as the guaranteed convergence rate, and the analysis process is complete.
\chapter{Results and Future Work}\label{chapter:Conclusion}

\section{Numerical result validation}

The \texttt{AlgorithmAnalysis.jl} package has been tested across a range of first-order optimization algorithms over the $m$-strong $L$-smooth convex function class to verify its ability to automatically generate accurate worst-case performance guarantees. In the test cases presented in this section, performance guarantees produced by the package matches to the mathematically produced results in \cite{tutorial}, proving that the package successfully implement Lyapunov-based algorithm analysis.

For every plots in this section, we set the value of $m$ to 1 and sample 12 values of $L$ that are logarithmically spaced between 1 and 100, using a base-10 scale. This ensures uniform coverage across orders of magnitude, capturing both small and large condition numbers with equal density in log space. The convergence rate produced by the package is plotted it on the y-axis, while the x-axis plots the condition number $L/m$ of the tested $(m,L)$ values.

\subsection*{Gradient descent}

Continuing the example in Figure \ref*{ex_analysis}, we plot the package's derived worst-case convergence rate guarantee of the gradient descent algorithm with step size $\alpha = 2/(L+m)$ at optimizing 12 different $m$-strong $L$-smooth convex function classes, using the 12 sampled $L$ values. The plot of the result produced is presented in Figure \ref*{gd_results}.

\begin{figure}[h]
    \centering
    \includegraphics[width = .9 \textwidth]{gd_results.pdf}
    \caption{Convergence rate guarantee of gradient descent over sector smooth strongly convex classes}
    \label{gd_results}
\end{figure}

We have also tested the same gradient descent algorithm over 12 $(m, L)$ sector-bounded function classes. The code used to find the worst-case convergence rate guarantees of the fast gradient algorithm at optimizing $(m, L)$ sector-bounded function classes is presented in Figure \ref*{gd_sectorbounded_code} and the plot of the result produced is presented in Figure \ref*{gd_sectorbounded_results}. We expect the analysis result to be identical to that of gradient descent over $m$-strong $L$-smooth convex function classes.

\begin{figure}[h!]
	\begin{lstlisting}[mathescape]
$\alpha$ = 2/(L+m)
@algorithm begin
    f = DifferentiableFunctional{R$^n$}()
    xs = first_order_stationary_point(f)
    f' $\in$ SectorBounded(m, L, xs, f'(xs))
    x0 = R$^n$()
    x1 = x0 - $\alpha$*f'(x0)
    x0 => x1
    performance = (x0-xs)^2
end
@show rate(performance)
\end{lstlisting}
\caption{Analysis of GD and $m$-$L$ sector bounded functions}
\label{gd_sectorbounded_code}
\end{figure}

\begin{figure}[h]
    \centering
    \includegraphics[width = .9 \textwidth]{gd_sectorbounded_results.pdf}
    \caption{Convergence rate guarantee of gradient descent over sector bounded function classes}
    \label{gd_sectorbounded_results}
\end{figure}
\subsection*{Fast gradient}

We tested the fast gradient algorithm with step sizes $\alpha = 4/(3L + m)$ and $\beta = (\sqrt{3L + 1} - 2)/(\sqrt{3L + 1} + 2)$. The code used to find the worst-case convergence rate guarantees of the fast gradient algorithm at optimizing $m$-strong $L$-smooth convex function classes is presented in Figure \ref*{fg_code} and the produced plot is presented in Figure \ref*{fg_results}.

\begin{figure}[h!]
	\begin{lstlisting}[mathescape]
$\alpha$ = 4/(3*L+m); $\beta$=(sqrt(3*L+1)-2)/(sqrt(3*L+1)+2)
@algorithm begin
    f = DifferentiableFunctional{R$^n$}()
    xs = first_order_stationary_point(f)
    f $\in$ SmoothStronglyConvex(m, L)
    x0 = R$^n$()
    x1 = R$^n$()
    y1 = x1 + $\beta$*(x1 - x0)
    x2 = y1 - $\alpha$*f'(y1)
    y2 = x2 + $\beta$*(x2 - x1)
    x3 = y2 - $\alpha$*f'(y2)
    x0 => x1
    x1 => x2
    x2 => x3
    performance = (y1-xs)^2
end
@show rate(performance)
\end{lstlisting}
\caption{Analysis of FG and $m$-smooth $L$-strongly convex functions}
\label{fg_code}
\end{figure}

\begin{figure}[h]
    \centering
    \includegraphics[width = .9 \textwidth]{fg_results.pdf}
    \caption{Convergence rate guarantee of fast gradient over smooth strongly convex function classes}
    \label{fg_results}
\end{figure}

\subsection*{Heavy ball}

We tested the heavy ball algorithm with the following step sizes:
\[
\alpha = \frac{4}{\left( \sqrt{L} + \sqrt{m} \right)^2}, \quad
\beta = \left( \frac{\sqrt{L/m} - 1}{\sqrt{L/m} + 1} \right)^2.
\]
The code used to find the worst-case convergence rate guarantees of the heavy ball algorithm for optimizing $m$-strongly convex and $L$-smooth function classes is presented in Figure~\ref{hb_code}, and the resulting plot is shown in Figure~\ref{hb_results}.


\begin{figure}[h!]
	\begin{lstlisting}[mathescape]
$\alpha$ = 4/((sqrt(L)+sqrt(m))^2); $\beta$=((sqrt(L/m)-1)/(sqrt(L/m)+1))^2
@algorithm begin
    f = DifferentiableFunctional{R$^n$}()
    xs = first_order_stationary_point(f)
    f $\in$ SmoothStronglyConvex(m, L)
    x0 = R$^n$()
    x1 = R$^n$()
    x2 = x1 - $\alpha$*f'(x1) + $\beta$*(x1-x0)
    x3 = x2 - $\alpha$*f'(x2) + $\beta$*(x2-x1)
    x0 => x1
    x1 => x2
    x2 => x3
    performance = (x0-xs)^2
end
@show rate(performance)
\end{lstlisting}
\caption{Analysis of HB and $m$-smooth $L$-strongly convex functions}
\label{hb_code}
\end{figure}

\begin{figure}[h]
    \centering
    \includegraphics[width = .9 \textwidth]{hb_results.pdf}
    \caption{Convergence rate guarantee of heavy ball over smooth strongly convex function classes}
    \label{hb_results}
\end{figure}

\subsection*{Triple momentum}

We tested the triple momentum algorithm, which was developed by Van Scoy, Freeman, and Lynch in \cite{TMM} to be the fastest known globally convergent first-order algorithm at optimizing strongly convex functions. The algorithm is defined as:
\begin{equation}\label{eqn:GD}
    x_{k+1}= (1+\beta)x_{k} - \beta x_{k-1} - \alpha \nabla f((1+\gamma)x_k - \gamma x_{k-1})
  \end{equation}

The triple momentum algorithm's optimal parameters is determined by the condition number \( k = L/m \) when optimizing $m$-smooth $L$-strongly convex functions. The algorithm's parameters are defined as:

\[
\begin{aligned}
\rho &= 1 - \frac{1}{\sqrt{k}}, \\
\alpha &= \frac{1 + \rho}{L}, \\
\beta &= \frac{\rho^2}{2 - \rho}, \\
\gamma &= \frac{\rho^2}{(1 + \rho)(2 - \rho)}, \\
% \delta &= \frac{\rho^2}{1 - \rho^2}.
\end{aligned}
\]
Under these parameters, it was proven in \cite{TMM} that the performance guarantee matches the function $1-\sqrt{m/L}$, which is also plotted in along side the rates produced by the package in Figure~\ref{tmm_results}. The code used to find the worst-case convergence rate guarantees of the triple momentum algorithm is presented in Figure~\ref{tmm_code}.

\begin{figure}[h!]
    \centering
    \includegraphics[width = .8 \textwidth]{tmm_results.pdf}
    \caption{Convergence rate guarantee of triple momentum over smooth strongly convex function classes}
    \label{hb_results}
\end{figure}

\begin{figure}[h!]
	\begin{lstlisting}[mathescape]
k = L/m
rho = 1 - 1/(sqrt(k))
$\alpha$ = (1 + rho)/L
$\beta$ = (rho^2)/(2-rho)
gamma = (rho^2)/((1+rho)*(2-rho))
delta = (rho^2)/(1-rho^2)
@algorithm begin
    f = DifferentiableFunctional{R$^n$}()
    xs = first_order_stationary_point(f)
    f $\in$ SmoothStronglyConvex(m, L)
    x0 = R$^n$()
    x1 = R$^n$()
    y1 = (1+gamma)*x1 - gamma*x0
    x2 = (1+$\beta$)*x1 - $\beta$*x0 - $\alpha$*f'(y1)
    y2 = (1+gamma)*x2 - gamma*x1
    x3 = (1+$\beta$)*x2 - $\beta$*x1 - $\alpha$*f'(y2)
    x0 => x1
    x1 => x2
    x2 => x3
    performance = ((1+delta)*x2 - delta*x1 -xs)^2
    
@show rate(performance)
\end{lstlisting}
\caption{Analysis of HB and $m$-smooth $L$-strongly convex functions}
\label{tmm_code}
\end{figure}

\section{Future work}

While the current experimental validations demonstrate that the \texttt{AlgorithmAnalysis.jl} package accurately generates worst-case performance guarantees for several prominent first-order optimization methods, we can explore other other methods to test compatibility. Future work should focus on systematically testing and verifying the package against a wider range of first-order unconstrained optimization algorithms. This extended analysis will further confirm the robustness of the package's implementation of the Lyapunov fucntion-based approach and highlight any potential limitations to be remedied and improved.

Additionally, the program currently does not support automatic ''lifting dimension'', a step of the Lyapunov method to algorithm analysis in \cite{tutorial}. While the user can manually add a lifting dimension by defining the analyze algorithm using more updates than the required 2. A function which can create these extra updated states, label them, and add them to the analysis process can allow the user to implement lifting dimension having only to enter how many extra updated states they wish to use. The lifting dimension will increase the size of the Lyapunov functions and introduce additional interpolated points and therefore additional constraints. As a result, the optimization problem we are solving to certify a convergence rate will becore more complexto the optimization problem while this research work is to implement this step and measure its effect on the tightness of the convergence rate guarantee.




%%%%%%%%%%%%%%%%%%%%%%%%%%%%%%%%%%%%%%%%%%%%%%%%%%%%%%%%%%%%%%%%%%%%%%%%%%%%%%%
%% APPENDICES
%%%%%%%%%%%%%%%%%%%%%%%%%%%%%%%%%%%%%%%%%%%%%%%%%%%%%%%%%%%%%%%%%%%%%%%%%%%%%%%
\appendices
% \input{chapters/A - background.tex}

\backmatter


%%%%%%%%%%%%%%%%%%%%%%%%%%%%%%%%%%%%%%%%%%%%%%%%%%%%%%%%%%%%%%%%%%%%%%%%%%%%%%%
%% BIBLIOGRAPHY
%%%%%%%%%%%%%%%%%%%%%%%%%%%%%%%%%%%%%%%%%%%%%%%%%%%%%%%%%%%%%%%%%%%%%%%%%%%%%%%
\addcontentsline{toc}{chapter}{References}
\renewcommand\bibname{References}
\bibliography{references}

\end{document}